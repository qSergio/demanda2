% !TEX TS-program = pdflatex
% !TEX encoding = UTF-8 Unicode

% This file is a template using the "beamer" package to create slides for a talk or presentation
% - Giving a talk on some subject.
% - The talk is between 15min and 45min long.
% - Style is ornate.

% MODIFIED by Jonathan Kew, 2008-07-06
% The header comments and encoding in this file were modified for inclusion with TeXworks.
% The content is otherwise unchanged from the original distributed with the beamer package.

\documentclass{beamer}


% Copyright 2004 by Till Tantau <tantau@users.sourceforge.net>.
%
% In principle, this file can be redistributed and/or modified under
% the terms of the GNU Public License, version 2.
%
% However, this file is supposed to be a template to be modified
% for your own needs. For this reason, if you use this file as a
% template and not specifically distribute it as part of a another
% package/program, I grant the extra permission to freely copy and
% modify this file as you see fit and even to delete this copyright
% notice. 


\mode<presentation>
{
  \usetheme{Default}
  % or ...
  \usecolortheme{dolphin}

  \setbeamercovered{transparent}
  % or whatever (possibly just delete it)
}

\usepackage[utf8]{inputenc}
\usepackage[english,spanish,activeacute]{babel}
% or whatever



% or whatever
%\usepackage[spanish,activeacute]{babel}
\usepackage{amsmath}
%\usepackage{amsfonts}%
%\usepackage{latexsym,amssymb}%


\usepackage{graphicx}
%\usepackage[T1]{fontenc} % if needed
\usepackage{physics}
\usepackage{epigraph}
\usepackage{tcolorbox}

\usepackage{times}

% Or whatever. Note that the encoding and the font should match. If T1
% does not look nice, try deleting the line with the fontenc.
%%%%%%%%%%%%%%%%%%%%%%%%%%%%%%%%%%%%%%
\usepackage{tikz}

\usepackage{tikz-cd}


%\usepackage{amsmath, amsthm, amssymb, amsfonts}




\setbeamertemplate{background canvas}{\begin{tikzpicture}\node[opacity=.3,inner sep=0]{\includegraphics
[width=\paperwidth,height=\paperheight]{datan_f.png}};\end{tikzpicture}}

%\pgfdeclareimage[width=\paperwidth]{mybackground}{titulo_ncg.PNG}

%\setbeamertemplate{title page}{

   %     \begin{picture}(0,0)

       %     \put(-30,-163){%
           %     \pgfuseimage{mybackground}
            %}

            %\put(0,-110.7){%
                %\begin{minipage}[b][45mm][t]{226mm}
                    %\usebeamerfont{title}{\inserttitle\par}
                %\end{minipage}
%            }

    %        \end{picture}

    %}
%%%%%%%%%%%%%%%%%%%%%%%%%%%%%%%5


\title % (optional, use only with long paper titles)
{}
%{Los Expedientes Secretos de la Geometría: el Mundo Cuántico}

%\subtitle
%{Presentation Subtitle} % (optional)

\author % (optional, use only with lots of authors)
{}
%{Sergio Nieto}
% - Use the \inst{?} command only if the authors have different
%   affiliation.

%\institute[Universidad Nacional Autónoma de México ] % (optional, but mostly needed)
%{ }
%\institute{} % (optional, but mostly needed)

% - Use the \inst command only if there are several affiliations.
% - Keep it simple, no one is interested in your street address.

\date {}
%\subject{Talks}
% This is only inserted into the PDF information catalog. Can be left
% out. 

%\titlegraphic{\includegraphics[width=\textwidth,height=.75\textheight]{titulo_ncg.PNG}}

% If you have a file called "university-logo-filename.xxx", where xxx
% is a graphic format that can be processed by latex or pdflatex,
% resp., then you can add a logo as follows:

% \pgfdeclareimage[height=0.5cm]{university-logo}{university-logo-filename}
% \logo{\pgfuseimage{university-logo}}



% Delete this, if you do not want the table of contents to pop up at
% the beginning of each subsection:
%\AtBeginSection[]
%{
  %\begin{frame}<beamer>{Plan}
    %\tableofcontents[currentsection,currentsubsection]
  %\end{frame}
%}


% If you wish to uncover everything in a step-wise fashion, uncomment
% the following command: 

%\beamerdefaultoverlayspecification{<+->}


\begin{document}

{
\usebackgroundtemplate{\includegraphics[width=\paperwidth,height=\paperheight]{NCG_titulo.PNG}}
\begin{frame}
\titlepage
\end{frame}
}


%\begin{frame}
  %\titlepage
%\end{frame}
%%%%%%%%%%%%%%%%%%%%%%%%%%%%%%%%%%%%%%

\begin{frame}{Plan}
  \tableofcontents
  % You might wish to add the option [pausesections]
\end{frame}


% Since this a solution template for a generic talk, very little can
% be said about how it should be structured. However, the talk length
% of between 15min and 45min and the theme suggest that you stick to
% the following rules:  

% - Exactly two or three sections (other than the summary).
% - At *most* three subsections per section.
% - Talk about 30s to 2min per frame. So there should be between about
%   15 and 30 frames, all told.

\section{Motivación}

%\subsection[Short First Subsection Name]{First Subsection Name}

\begin{frame}{El Modelo Estándar}
\begin{block}{}
Los ingredientes básicos de nuestro universo están descritos en el \textit{Modelo Estándar de la Física de Partículas}.
\end{block}
  \vspace{5mm}
  \begin{columns}
\column{.4\textwidth}
\includegraphics[height=5cm]{lagrangian.jpg}
\column{.5\textwidth}
\includegraphics[height=5cm]{Standard3.png}
\end{columns}

\end{frame}

\begin{frame}{El Modelo Estándar}
Mecánica Cuántica $+$ Relatividad Especial $=$ TCC (QFT).
\vspace{5mm}
 \begin{itemize}
  \item
    En una TCC tenemos campos y las exitaciones localizadas de éstos nos dan las partículas.
    \begin{figure}
\includegraphics[height=5cm]{water.jpg}
\end{figure}
  \end{itemize}
\end{frame}

\begin{frame}{El Modelo Estándar}
%Mecánica Cuántica $+$ Relatividad Especial $=$ TCC (QFT).
%\vspace{5mm}
 \begin{itemize}
  \item
    Bosones, mediadores de las interacciones (espín entero) y Fermiones, que corresponden a los campos de materia (espín semi-entero).
   \item 
   Describe 3 de las interacciones conocidas: Fuerte, Electro-Débil.
   \item
   Además tenemos al Campo de Higgs (espín $= 0$).
\begin{figure}
\includegraphics[height=3.6cm]{higgs.png}
\end{figure}
  \end{itemize}
\end{frame}

\begin{frame}{El Modelo Estándar}
 \begin{itemize}
  \item
    Todo lo que entendemos acerca de las leyes básicas de la naturaleza proviene de las \textit{simetrías}.
   \item 
   Hay {\bf{Simetrías del Espaciotiempo}}: Invariancia de Lorentz, Poincaré, Difeomorfismos.
   \item
   {\bf{Simetrías Internas}}: transformaciones sobre los campos mismos.
   \begin{exampleblock}{}
\[
G_{MS} = SU(3)_c \times SU(2)_L \times U(1)_Y  
\]
\end{exampleblock}

  \end{itemize}
\end{frame}

\begin{frame}{}
 \begin{itemize}
   \item
   {\bf{Modelo Lineal}}: se asume que la distribución de la variable que deseamos predecir cumple que 
 \begin{exampleblock}{}
\[
Y|X \thicksim \mathcal{N}(\mu(X) , \sigma^2 I)
\]
\end{exampleblock}

\begin{exampleblock}{}
\[
\mathbb{E}(Y|X) = \mu(X) = X^T \beta  
\]
\end{exampleblock}

  \end{itemize}
\end{frame}


\begin{frame}{}

 \begin{itemize}
   \item
   {\bf{Modelo Lineal}}: se asume que la distribución de la variable que deseamos predecir cumple que 
 \begin{exampleblock}{}
\[
Y|X \thicksim \mathcal{N}(\mu(X) , \sigma^2 I)
\]
\end{exampleblock}

\begin{exampleblock}{}
\[
\mathbb{E}(Y|X) = \mu(X) = X^T \beta  
\]
\end{exampleblock}

  \end{itemize}
\end{frame}


\begin{frame}{}
 \begin{itemize}
   \item
   {\bf{Generalización}}: podemos usar un modelo general lineal es dos formas (no excluyentes). 
\begin{enumerate}
\item Generalizamos la distribución de $Y$:
 \begin{exampleblock}{}
\[
Y \thicksim \text{distribución de la familia exponencial}
\]
\end{exampleblock}

\item Generalizamos la relación en las covariables y la variable objetivo:
\begin{exampleblock}{}
\[
g(\mu(X)) = X^T \beta,  \, \text{donde} \, \mu = \mathbb{E}(Y|X).   
\]
\end{exampleblock}
 $g$ se conoce como función de {\it{enlace}} o {\it{link}}.
\end{enumerate}
  \end{itemize}
\end{frame}

\begin{frame}{}
 \begin{itemize}
   \item
   {\bf{Modelo Estacional}}: vamos a construir un modelo lineal para una serie de tiempo, pero sin auto-regresión, será únicamente estacional. \\

Sea $Y_t$ una serie de tiempo con $t \in \{  1,2,3, \dots \}$ tal que: 
 \begin{exampleblock}{}
\[
Y_t \thicksim \mathcal{N}( \mu_t , \sigma^2 I)
\]
\end{exampleblock}

  \end{itemize}
\end{frame}

\begin{frame}{}
 \begin{itemize}
   \item
   {\bf{Modelo Estacional}}: La idea es usar los eventos estacionales como la semana, el año y el semestre. \\

Definimos la media $\mu_t$ en función de variables únicamente categóricas: 
\begin{exampleblock}{}
\[
 \mu_t = \alpha +  \beta_{\text{año}_t}  + \gamma_{\text{semana}_t} + \epsilon_{\text{semestre}_t} +  (\beta \gamma)_t + ( \beta \epsilon)_t + (\gamma \epsilon)_t
\]
\end{exampleblock}

donde, $\textit{año}_t \in \{ 1, 2\}$,  $\textit{semestre}_t \in \{ 1, 2\}$ y $\textit{semana}_t \in \{ 1, 2, \dots , 26\}$.    

  \end{itemize}
\end{frame}



\begin{frame}{El Modelo Estándar}
 \begin{itemize}
  \item
    Las \textit{simetrías} determinan las interacciones.
   \end{itemize} 
    \begin{exampleblock}{}
     Las {\textit{simetrías globales}} se pueden obligar a ser \textit{locales}: 
\[
\mathcal{L} = \partial_{\mu}\Phi \partial_{\mu}\Phi^* - V(\Phi \Phi^*)
\]
Es invariante bajo la transformación $\Phi \to e^{i\alpha} \Phi$. 
Si pedimos que $\alpha = \alpha(x)$, tenemos que $\partial_{\mu}\Phi \to e^{i \alpha(x)} (\partial_{\mu}\Phi +  i \partial_{\mu} \alpha \Phi)$.
\end{exampleblock}
 \begin{exampleblock}{}
Para recuperar la invariancia debemos definir:
\[
D_{\mu}\Phi = \partial_{\mu} \Phi +  i A_{\mu} \Phi
\]
Y pedir que transforme a nuestro favor: $A'_{\mu} = A_{\mu} - \partial_{\mu} \alpha $ 
\end{exampleblock}
\end{frame}


\begin{frame}{El Modelo Estándar}

\begin{itemize}
  \item
    La introducción de un nuevo campo $A_{\mu}$ nos permite hacer local la simetría (otros ejemplos $\mathcal{L}_{Dirac} = \bar{\psi} \gamma^{\mu} D_{\mu} \psi$). 
    \item Este nuevo campo es mediador de una intearcción que antes no estaba ahí, p. ej.: $A_{\mu}\Phi A_{\mu} \Phi^*$, $\bar{\psi} A_{\mu} \psi$, etc.
    \item A veces las simetrías están escondidas:
\begin{figure}
\includegraphics[height=4cm]{mexican.png}
\end{figure}
   \end{itemize} 
\end{frame}

\begin{frame}{Más allá del Modelo Estándar...}

\begin{itemize}
  \item
   El MS tiene algunos problemas: no incluye a la gravedad, no explica $G_{MS}$, no explica las dimensiones $3+1$, no sabemos si habrá más partículas...
  
    \item Cómo vamos más allá?
    \begin{exampleblock}{}
i) Más simetrías internas (TGU)

\[
G \to G_{MS} \to SU(3) \times U(1)
\]

ii) Más simetrías del espaciotiempo $\to$ más dimensiones, SUSY. 
 
\end{exampleblock}
   \end{itemize} 
\end{frame}

\begin{frame}{Más allá del Modelo Estándar...}

\begin{itemize}
   \item Los infinitos debajo de la alfombra...
   \vspace{5mm}
   \item La Teoría Cuántica de Campos contiene singularidades: divergencias ultravioleta, etc.
   \vspace{5mm}
   \item A pesar de que se sabe lidiar con estas vía un esquema de Renormalización, los resultados son sólo perturbativos.
   \vspace{5mm}
\item 
Quizá necesitamos una nueva noción de espaciotiempo...
   \end{itemize} 
\end{frame}


\begin{frame}{Relatividad General}

\begin{exampleblock}{}
La geometría de nuestro espaciotiempo está codificada dentro de la Teoría de Relatividad General.
\[
R_{\mu \nu} - \frac{1}{2} R g_{\mu \nu} + \Lambda g_{\mu \nu}= 8 \pi T_{\mu \nu} 
\]
\end{exampleblock}
\begin{figure}
\includegraphics[height=4.5cm]{hoyonegro.jpg}
\end{figure}
\end{frame}

\begin{frame}{Relatividad General}

 \begin{itemize}
  \item El espaciotiempo es una colección de eventos físicos.
\item
 Los puntos de la geometría pierden importancia física debido a la invariancia bajo $Diff(M)$.
  \item
  Contiene singularidades: hoyos negros, el big bang...


  \end{itemize}
  
\begin{figure}
\includegraphics[height=6.5cm]{boardinterstellar.png}
\end{figure}
\end{frame}

\begin{frame}{Gravedad Cuántica}
Si tratamos de combinar la RG y la Teoría Cuántica de Campos... 
\begin{figure}
\includegraphics[height=4.5cm,width=11.5cm]{QGgarbage.png}
\end{figure}

\end{frame}



\begin{frame}{Esto no puede ser todo...}
\begin{itemize}
\item A escalas suficientemente pequeñas, la escala de Planck, las nociones clásicas de geometría ya no pueden describir propiamente nuestro espaciotiempo. 
\item Cualquier intento para medir propiedades físicas en una región del tamaño de la longitud de Planck necesita tanta energía que fluctuaciones muy grandes en la métrica del espaciotiempo ocurrirían...
\begin{exampleblock}{}
\[
\l_{p} = \sqrt{\frac{\hbar G}{c^3}} = 1.616 \times 10^{-35} m
\]
\end{exampleblock}
\pause
\item Nuevamente, necesitamos nuevas ideas.
\end{itemize}
\end{frame}


\section{Geometría No Conmutativa}

\begin{frame}{Geometría Cuántica}
\begin{itemize}
\item Una alternativa natural es usar Geometría Cuántica, también llamada {\bf{Geometría No Conmutativa}}. 
\vspace{5mm}
\item Con esta herramienta se pueden modificar e incluso extender la geometría y las propiedades del espaciotiempo.
\vspace{5mm}
 \item La Geometría No Conmutativa se originó justamente en el contexto de  la Mecánica Cuántica. 
 
 \end{itemize}
\end{frame}


\begin{frame}{Repaso de Mecánica Cuántica}
\begin{itemize}


\item Denotamos los estados como \textit{kets}: $\lvert \psi \rangle$.
\vspace{5mm}

\item También un estado posible es la superposición: $\alpha \lvert \psi_1 \rangle + \beta \lvert \psi_2 \rangle = \lvert \alpha \psi_1 + \beta \psi_2 \rangle$.
\vspace{5mm}

 \item Es decir, los estados del sistema forman un espacio vectorial. 
\vspace{5mm}
\item  En este espacio vectorial hay un producto escalar: $\langle \phi \lvert \psi \rangle \in \mathbb{C}$.

\end{itemize}
\end{frame}

\begin{frame}{Repaso de Mecánica Cuántica}
\begin{itemize}

\item Respecto a la \textit{norma} $\lvert \lvert \psi \rvert \rvert \equiv \sqrt{\langle \psi \lvert \psi \rangle}$ supondremos que este espacio vectorial es un espacio de Hilbert, $\mathcal{H}$.
\vspace{5mm}

\item Hay una noción de \textit{vector dual} a un ket: el \textit{bra} $\langle \psi \lvert$, toma un ket y lo mapea a los complejos $\mathbb{C}$ a través del producto escalar.
\vspace{5mm}
\item Por ejemplo, para el vector $ \lvert \psi \rangle = \begin{pmatrix}
a  \\
b  \\
\end{pmatrix}$, el dual es su conjugado traspuesto:


$ \langle \psi \lvert \psi \rangle = 
(\bar{a} \, \bar{b})
\begin{pmatrix}
a  \\
b  \\
\end{pmatrix} = a^2 + b^2.
$
\end{itemize}
\end{frame}


\begin{frame}{Mecánica Cuántica}

\begin{itemize}

\item En Mecánica Cuántica los observables se promueven a operadores lineales $f: \mathcal{H} \to \mathcal{H}$.
\vspace{5mm}
\item Los valores propios de este operador son los únicos resultados posibles de una medición (idealizada) de $O$.
\vspace{5mm}
\item Decimos que las mediciones de $O$ y $O'$ no interfieren una con otra solo si $[O,O' ] = 0$.
\vspace{5mm}
\item Los observables corresponden a operadores hermitianos $O^{*} = O$, lo que implica que tienen valores propios reales.

\end{itemize}
\end{frame}



\begin{frame}{Mecánica Cuántica}

\begin{itemize}

\item Para una partícula, las observables básicas son las posición y el momento: $x^i, p^j$. Y se promueven a operadores bajo cuantización $\hat{x}, \hat{p}$.

\vspace{5mm}
\item Las relaciones de conmutación de Heisenberg son: $[\hat{x}, \hat{p} ] = i \hbar \delta^{ij}$.
\vspace{5mm}
\item Después de la cuantización, estos operadores ahora forman parte de un \textit{álgebra no conmutativa}.
\vspace{5mm}
\item Si introducimos $u = e^{ix}$ y $v = e^{ip}$. El álgebra entera es generada por estos dos elemntos y la relación $uv = qvu$ donde $q = e^{- i \hbar}$.
\end{itemize}
\end{frame}



\begin{frame}{Geometría No Conmutativa}

\begin{itemize}

\item Lo anterior es un ejemplo del plan general de la Geometría No Conmutativa.

\vspace{5mm}
\item El propósito de la GNC es reformular tanto como sea posible la geometría de un espacio o variedad en términos del álgebra de funciones definida sobre este espacio.
\vspace{5mm}
\item La principal noción que se pierde en el paso de lo clásico a lo cuántico para una partícula es la localización, la noción de \textit{punto}.
\vspace{5mm}
\item El formalismo de esta \textit{Geometría Cuántica} consiste en juntar conceptos clásicos de geometría diferencial con álgebras no conmutativas y análisis funcional.
\end{itemize}
\end{frame}



\begin{frame}{Conceptos Básicos en GNC}

\begin{itemize}

\item Una $*$-{\bf{álgebra}} $\mathcal{A}$ es un espacio vectorial sobre $\mathbb{C}$ con una multiplicación $\cdot : \mathcal{A} \to \mathcal{A} $ y una involución $* : \mathcal{A} \to \mathcal{A}$:
\begin{exampleblock}{}
\[
(a + b) \cdot c = ac + bc
\]
\[
a\cdot (b+c) = ab + ac
\]
\[
a(bc) = (ab)c
\]
\[
\lambda (ab) = (\lambda a)b = a (\lambda b), \, \lambda \in \mathbb{C}
\]
\[
(\lambda a+b)^* = \bar{\lambda} a^* + b^*
\]
\[
(ab)^* = b^* a^*
\]
\[
(a^*)^* = a
\]
\end{exampleblock}
\end{itemize}
\end{frame}



\begin{frame}{Conceptos Básicos en GNC}

\begin{itemize}
\item Decimos que $\mathcal{A}$ es una álgebra normada si $\mathcal{A}$ es un espacio normado tal que $\| ab \| \leq \| a \| \| b \|$.
\vspace{5mm}
\item Decimos que $\mathcal{A}$ es una álgebra normada de Banach si $\mathcal{A}$ es un espacio normado y completo bajo esa norma.
\vspace{5mm}
\item Una {\bf{álgebra}} $C^*$ es una $*$-álgebra de Banach donde se cumple que:
\begin{exampleblock}{}
\[
\|aa\|^* = \| a \|^2
\]
\end{exampleblock}
\end{itemize}
\end{frame}



\begin{frame}{Puntos, qué son?}

\begin{itemize}
\item En geometría clásica identificamos puntos por medio de coordenadas. Estas son funciones definidas sobre el esacio.
\vspace{5mm}
\item La elección de las funciones que nos permiten distinguir puntos es parte de la Topología. Podemos usar funciones continuas a los complejos. Estas forman una $*$-álgebra.
\vspace{5mm}
\item De hecho,si $X$ es un espacio topológico (localmente) compacto tipo Hausdorff y $C(X)$ denota al álgebra de funciones continuas sobre $X$, entonces $C(X)$ es una álgebra $C^*$ (no) unital y conmutativa.
\end{itemize}
\end{frame}


\begin{frame}{Puntos, qué son?}

\begin{itemize}
\item Necesitamos aprender dos teoremas muy importantes que establecen el marco de lo que queremos explicar.
\vspace{5mm}
\end{itemize}
\begin{block}{Teorema de Gelfand-Naimark (1943)}
Si una álgebra $C^*$, $\mathcal{A}$, es conmutativa, entonces es un álgebra de funciones continuas sobre algún espacio topológico.
\end{block}
\begin{block}{Teorema de Gelfand-Naimark-Segal (1947)}
Toda álgebra $C^*$, $\mathcal{A}$, es isométricamente isomorfa a una específica $C^*$ álgebra de operadores sobre un espacio de Hilbert.
\end{block}
\end{frame}


\begin{frame}{Puntos, qué son?}

\begin{itemize}
\item Estos resultados nos dicen que toda $C^*$ álgebra es un álgebra de operadores acotados sobre un espacio de Hilbert y que no perdemos nada al pasar de una descripción a otra.
\vspace{5mm}
\item Los puntos de un espacio se pueden caracterizar entonces de manera puramente algebraica. Para un álgebra $C^*$ conmutativa que corresponde a $C(X)$, que $x \in X$ sea un punto es lo mismo a que evalúe funciones,  $\kappa_x : \mathcal{A} \to \mathbb{C}, \kappa_x (f) = f(x) $.
\vspace{5mm}
\item Estas funciones se llaman caracteres y son lineales, multiplicativas, normalizadas y hermitianas.
\end{itemize}
\end{frame}



\begin{frame}{Ejemplo: El Plano Cuántico}

\begin{itemize}
\item Por el principio de incertidumbre de Heisenberg es imposible medir simultáneamente la posición y el momento con precisión arbitraria.
\vspace{5mm}
\item En la teoría cuántica el espacio fase de coordenadas $x,p$ ya no puede ser un espacio en el sentido clásico.
\vspace{5mm}

\[
\kappa (pq) - \kappa (qp) = i \hbar 
\]
\[
\implies \kappa (p) \kappa(q) - \kappa (q)\kappa (p) = i \hbar 
\]
\[
\implies 0 = i \hbar !
\]

\end{itemize}
\end{frame}




\begin{frame}{Ejemplo: El Plano Cuántico}
\begin{itemize}
\item Todo lo que nos queda es un álgebra no conmutativa de funciones definida sobre un plano sin puntos, un \textit{plano cuántico}.

\end{itemize}

\begin{figure}
\includegraphics[height=4.5cm]{fuzzyplane.png}
\end{figure}
\end{frame}


\begin{frame}{Ejemplo: La Esfera Cuántica}

\begin{itemize}
\item Vamosa fijarnos en el álbegra de matrices $M_2 (\mathbb{C})$. Esta álgebra es no conmutativa y se puede escribir una base que la genera.
\vspace{5mm}
\item Esta base la forman la matriz identidad y las matrices de Pauli:

\[
\sigma_1 =
  \begin{bmatrix}
    0 & 1  \\
    1 & 0
  \end{bmatrix}, \,
\sigma_2 =
  \begin{bmatrix}
    0 & -i  \\
    i & 0
  \end{bmatrix}, \,
  \sigma_3 =
  \begin{bmatrix}
    1 & 0  \\
    0 & -1
  \end{bmatrix}, \,  
\]
\vspace{5mm}
\item Cumplen las relaciones $\sigma_i \sigma_j +  \sigma_j \sigma_i = 2 \delta_{ij}$ que reflejan la no conmutatividad.
\end{itemize}
\end{frame}


\begin{frame}{Ejemplo: La Esfera Cuántica}

\begin{itemize}
\item Las matrices de Pauli se pueden poner en correspondencia con los cuaterniones $\mathbb{H}$. 
\[ \sigma_1 \to -i \sigma_1 = I 
\]
\[ \sigma_1 \to -i \sigma_2 = J 
\]
\[ \sigma_1 \to -i \sigma_3 = K 
\]

\vspace{5mm}
\item La estructura del álgebra de los cuaterniones es la misma que la de $SU(2)$.

\item Así como los espacios geométricos tienen simetrías que los caracterizan, las álgebras tienen \textit{Automorfismos}. 

\end{itemize}
\end{frame}




\begin{frame}{Ejemplo: La Esfera Cuántica}

\begin{itemize}

\vspace{5mm}
\item De hecho se tiene el siguiente resultado:

\[
Aut(M_2 (\mathbb{C})) \cong SU(2)/_{\mathbb{Z}_2} \cong SO(3)
\]
\vspace{2mm}

\item Es por esta razón que al álgebra $M_2(\mathbb{C})$ le llamaremos \textit{esféra cuántica}.
\end{itemize}

\begin{figure}
\includegraphics[height=4.5cm]{fuzzys.png}
\end{figure}
\end{frame}
%%%%%%%%%%%%%%%%%%%%%%%%%%%%%%%%%%%
\begin{frame}{Será esto sólo ciencia ficción...}
\begin{figure}
\includegraphics[height=7.5cm]{amazing.png}
\end{figure}
\end{frame}

\section{Aplicaciones a la Física}

\begin{frame}{Geometría No Conmutativa à la Connes}

\begin{itemize}
\item Hay varias maneras de formular la Geometría No Conmutativa.
\vspace{5mm}
\item El programa propuesto por A. Connes es el siguiente
 \begin{itemize}
 \item Usar el álgebra de funciones definida sobre el espacio.
 \item Después, usar una manera de medir distancias o tamaños, será necesario derivar...
 \item Usar el espacio de Hilbert asociado para recuperar la geometría.
 \end{itemize} 

\vspace{5mm}
\item Estas 3 herramientas se denominan \textit{Triples Espectrales} y determinan por completo algunos espacios geométricos importantes, por ejemplo {\bf{el Modelo Estándar}}. 

\end{itemize}
\end{frame}



\begin{frame}{Geometría No Conmutativa à la Connes}

\begin{itemize}
\item Un triple espectral $(\mathcal{A}, \mathcal{H}, D)$ es por ejemplo:
\vspace{4mm}
 \begin{itemize}
 \item Un álgebra $\mathcal{A}$ de funciones sobre el espaciotiempo $M$.
 \item Un espacio de Hilbert $\mathcal{H}$ (funciones escalare, espinoriales...).
 \item El operador de Dirac $D_M = i \gamma^\mu \partial_\mu$
 \end{itemize} 
\vspace{6mm}
\item Los valores propios del operador $D$ son $\sim j^{1/dim(M)}$
 

\end{itemize}
\end{frame}




\begin{frame}{Podemos escuchar la Dimensión de un Espacio Geométrico?}

\begin{itemize}
\item Un triple espectral $(\mathcal{A}, \mathcal{H}, D)$ trata de responder a la vieja pregunta: \textit{podemos escuchar la forma del tambor?}
\vspace{5mm}
\item Todo espacio o variedad viene con un operador naturalmente asociado:
\[ \Delta = - \sum_{i} \frac{\partial^2}{\partial x_{1}^{2}}
\]
\vspace{2mm}
 \item H. Weyl probó que si sabíamos $N(\lambda) = \# \{ \lambda_i \leq \lambda \}$
 \[ \lim_{\lambda \to \infty} \frac{N(\lambda)}{\lambda} = \frac{A(M)}{4 \pi}
\]

\end{itemize}
\end{frame}


\begin{frame}{Podemos escuchar la Dimensión de un Espacio Geométrico?}


\begin{itemize}

\item Solo para algunas simples formas podemos conocer el espectro explícitametne. 
\end{itemize}
\begin{columns}
\column{.4\textwidth}

\begin{figure}
\includegraphics[height=5cm]{cuerdaspec.png}
\end{figure}

\column{.5\textwidth}

\begin{figure}
\includegraphics[height=4.5cm]{disk.png}
\end{figure}

\end{columns}
\end{frame}




\begin{frame}{La Acción Espectral}

\begin{itemize}
\item Usando el triple espectral se construye la \textit{acción espectral}. $S_{\Lambda}(D) =$ número de eigenvalores de D hasta un cierto corte UV $\Lambda$:
\[
S_{\Lambda}(D) = Traza(f (D/ \Lambda )) 
\]
\[
= \sum_{k \geq 0} f_k a_k(D^2 / \Lambda^2)
\]
\vspace{5mm}
\item Lo que A. Connes encontró fue que esta acción contiene a la acción del Modelo Estándar usando el álgebra:
 \[
\mathcal{A} = \mathbb{C} \oplus M_2 (\mathbb{C}) \oplus M_3 (\mathbb{C}) (\oplus \mathbb{C})
\]

\end{itemize}
\end{frame}



\begin{frame}{El Modelo Estándar}

\begin{itemize}
\item Recuperar la acción del modelo estándar requiere que se cumplan las siguientes relaciones
\[ \frac{g_{3}^{2} f_0}{2 \pi^2} = \frac{1}{4}, \, g_{3}^{2} = g_{2}^{2} = \frac{5}{3} g_{1}^{2} 
\] 
\begin{figure}
\includegraphics[height=5cm]{gut.png}
\end{figure}

\end{itemize}
\end{frame}



\begin{frame}{El Modelo Estándar}

\begin{itemize}
\item La forma geométrica que toma el álgebra sucede en la escala del corte $\Lambda$. Esta sería la escala de gran unificación.

\item GNC predice que debe ser la escala de Planck.
\vspace{3mm}
\item También se obtuvo una predicción para la masa del bosón de Higgs: $m_H \sim 168 GeV$, que está mal!
\vspace{3mm}
\item Esto se arregla introduciendo un nuevo campo escalar que interactúa con el campo de Higgs:
\[ V(\sigma , h) = \alpha_1 (h^2 + \sigma^2) + \alpha_2 h^4 + \alpha_3 h^2 \sigma^2 + \alpha_4 \sigma^4
\]
\item Esto permite $m_H \sim 125 GeV$ y que $m_\sigma = 10^12 GeV$
\end{itemize}
\end{frame}


\section{Conclusiones}

\begin{frame}{Conclusiones}

  % Keep the summary *very short*.
  \begin{itemize}
  \item
    Podemos hacer compatibles la Geometría y la Física aún a nivel NO cuántico (\alert{no estamos cuantizando la gravedad!}). Sin embargo podemos hacer predicciones que se pueden contrastarcon el experimento.
    \vspace{5mm}
  \item
    Otenemos un diccionario que traduce entre objetos geométricos clásico y cuánticos.
   
  \end{itemize}
  
 
\end{frame}

\begin{frame}{Conclusiones}
 \begin{figure}
\includegraphics[height=7.5cm]{diccionario2.png}
\end{figure}
\end{frame}

\begin{frame}
MUCHAS GRACIAS!

\end{frame}

\end{document}


